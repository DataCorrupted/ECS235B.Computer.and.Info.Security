\documentclass{article}

\title{ECS 235B Computer \& Info Security \\ Homework 2}
\author{Yuyang(Peter) Rong \\917781535 \\ PtrRong@ucdavis.edu}

\usepackage[utf8]{inputenc}
\usepackage{graphicx}
\usepackage[colorlinks,linkcolor=red]{hyperref}
\usepackage{amsmath, amsthm, amssymb}
\usepackage[shortlabels]{enumitem}
\usepackage{subfloat}
\usepackage{booktabs}
\usepackage{color}
\definecolor{mygreen}{rgb}{0,0.6,0}
\definecolor{mygray}{rgb}{0.5,0.5,0.5}
\definecolor{mymauve}{rgb}{0.58,0,0.82}

\usepackage{algorithm}
\usepackage{algpseudocode}
\providecommand{\algorithmautorefname}{Algorithm}


% For listings
% In case we need rust format, check https://github.com/denki/listings-rust
\usepackage{listings}
\lstset{ %
    frame=single,
    % numbers=left,
    backgroundcolor=\color{white},   % choose the background color
    basicstyle=\ttfamily\footnotesize,        % size of fonts used for the code
    breaklines=true,                 % automatic line breaking only at whitespace
    captionpos=b,                    % sets the caption-position to bottom
    commentstyle=\color{mygreen},    % comment style
    escapeinside={(*@}{@*)},         % if you want to add LaTeX within your code
    keywordstyle=\color{blue},       % keyword style
    stringstyle=\color{mymauve},     % string literal style
    tabsize=1,
    % where to put the line-numbers; possible values are (none, left, right)
    numbers = left,
    % how far the line-numbers are from the code
    numbersep = 10 pt,
    % the style that is used for the line-numbers
    numberstyle = \ttfamily,
    % the step between two line-numbers. If it's 1, each line will be numbered 
    stepnumber = 1,
}
\usepackage{ulem}
\usepackage{mathtools}
\begin{document}
\maketitle

\section*{Extra Credit}
(30 points) Generalize the give-read rule (in section 5.2.4.2) to give-write and give-append, and prove they preserve the simple security condition, the *-property, and the discretionary security property.

\subsubsection*{Solution}

Represent give-write request ad $r=(s_1, \text{give}, s_2, o, \text{w})$.
Say the current state of the system is $v = (b, m, f, h)$, then give-write rule should be $\rho_7(r, v)$:
\begin{algorithm}[t]
    \caption{Type inference} \label{alg:typeInference}
    \begin{algorithmic}
        \If{($r \notin \Delta(\rho_7)$)}
            \State $\rho_7(\text{illegal}, v) = (\text{illegal}, v)$
        \Else \If{($o \ne \text{root}(o) \text{ and } \text{parent}(o) \ne \text{root}(o) \text{ and } \text{parent}(o) \in b(s_1: \text{w})$) or ($\text{parent}(o) = \text{root}(o) \text{ and } \text{canallow}(s_1, o, v)$) or ($o = \text{root}(o) \text{ and } \text{canallow}(s_1, \text{root}(o), v)$)}
                \State $\rho_7(r, v) = (\text{yes}, (b, m \wedge m[s2, o] \leftarrow \text{w}, f, h))$
            \Else 
                \State  $\rho_7(r, v) = (\text{no}, v)$
            \EndIf
        \EndIf
    \end{algorithmic}
\end{algorithm}

\begin{proof}
    Suppose $v$ satisfies the simple security condition, the *-property, and the ds-property.
    Let $\rho_7(r, v) = (d, v')$. 
    Either $v' = v$ or $v' = (b, m \wedge m[s2, o] \leftarrow \text{w}, f, h)$.
    
    If $v' = v$, then $v'$ satisfies all properties.

    If $v' = (b, m \wedge m[s2, o] \leftarrow \text{w}, f, h) = (b', m', f', h')$. 
    Then we have $b = b' \rightarrow b' \subset b$
    Therefore, using Theorem 5.15, $b' \subset b, f = f'$(Condition 1, 2), $b' \subset b, \forall s, o, m[s, o] \subset m'[s, o]$(Condition 3).
    We can conclude that $v'$ also satisfies all properties.
\end{proof}
\end{document}
\grid
