\section*{Problem 2}

(30 points) In the Janus system, when the framework disallows a system call, the error code EINTR (interrupted system call) is returned.

\begin{enumerate}
    \item When some programs have read or write system calls terminated with this error, they retry the calls. 
    What problems might this create?
    \item Why did the developers of Janus not devise a new error code (say, EJAN) to indicate an unauthorized system call?    
\end{enumerate}

\subsubsection*{Solution}

Part of this answer is inspired by the original work: \href{https://www2.eecs.berkeley.edu/Pubs/TechRpts/1999/CSD-99-1056.pdf}{Janus: an approach for connement of untrusted applications}.

\begin{enumerate}
    \item It may create an infinite loop.
    
    Some programs may have the logic that if a system call fails, keep calling until it's successful.
    Suppose the program is not privilaged to do that system call, this is an infinite loop that never ends.

    To fix this, the system may heuristically add an upper bound, i.e. when a system call is used and failed in a continous X times, the OS may kill the process.

    \item Solaris does not support that behavior.
    
    This feature need the syscall handler to check the privilage of the caller, which adds complexity to system design.
\end{enumerate}