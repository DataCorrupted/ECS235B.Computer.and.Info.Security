
\section*{Problem 1}

(30 points) Consider the rule of transitive confinement. 
Suppose a process needs to execute a subprocess in such a way that the child can access exactly two files, one only for reading and one only for writing.

\begin{enumerate}
    \item Could capability lists be used to implement this? If so, how?
    \item Could access control lists implement this? If so, how?
\end{enumerate}

\subsubsection*{Solution}

\begin{enumerate}
    \item Capability lists can't be used. 
    
    Because to prevent a covert chanel, the parent has to forbit ALL other processes from reading the file, which is impossible.

    \item Access control lists can be used too.
    
    The parent process should add child's pid into the files access control lists first.
    It has to be parent process since child process don't have the privilage.
    And after the child dies the parent needs to maintain the access control list properly, i.e. delete that pid from the lists.

    It would be something like:

    \begin{align*}
         acl(file\_to\_read) & = \{ (subprocess, \{r\}) \} \\
         acl(file\_to\_write) & = \{ (subprocess, \{w\}) \} \\
         acl(subprocess) & = \{ (process, \{own\}), (subprocess, \{r, w, x, own\}) \} \\
         acl(process) & = \{ (process, \{r, w, x, own\}) \} \\
    \end{align*}
\end{enumerate}