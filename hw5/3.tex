\section*{Problem 3}

(40 points) Kernighan and Plauger argue a minimalist philosophy of tool building. 
Their thesis is that each program should perform exactly one task, and more complex programs should be formed by combining simpler programs. 
Discuss how this philosophy fits in with the principle of economy of mechanism. 
In particular, how does the advantage of the simplicity of each component of a software system offset the disadvantage of a multiplicity of interfaces among the various components?

\subsubsection*{Solution}

Simplicity of each component decreases the \textit{surface of attack}.

When there is a bug in the system, there can be only two locations: the component it self, or the way the system is composed.
The maintainer can simply take the system apart and test each component to determine if it is a component's fault or the system composition.

Furthermore, components designed with only ONE interface is easy to write, test, and maintain. 
Therefore, it is not likely to have a bug in the component. 

On the otherhand, components with multiple interfaces may have bugs dependent on the input.
Thus it is harder to isolate the problem.
