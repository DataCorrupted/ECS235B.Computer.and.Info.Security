\documentclass{article}

\title{ECS 235B Computer \& Info Security \\ Homework 4}
\author{Yuyang(Peter) Rong \\917781535 \\ PtrRong@ucdavis.edu}

\usepackage[utf8]{inputenc}
\usepackage{graphicx}
\usepackage[colorlinks,linkcolor=red]{hyperref}
\usepackage{amsmath, amsthm, amssymb}
\usepackage[shortlabels]{enumitem}
\usepackage{subfloat}
\usepackage{booktabs}
\usepackage{color}
\definecolor{mygreen}{rgb}{0,0.6,0}
\definecolor{mygray}{rgb}{0.5,0.5,0.5}
\definecolor{mymauve}{rgb}{0.58,0,0.82}

\usepackage{algorithm}
\usepackage{algpseudocode}
\providecommand{\algorithmautorefname}{Algorithm}


% For listings
% In case we need rust format, check https://github.com/denki/listings-rust
\usepackage{listings}
\lstset{ %
    frame=single,
    % numbers=left,
    backgroundcolor=\color{white},   % choose the background color
    basicstyle=\ttfamily\footnotesize,        % size of fonts used for the code
    breaklines=true,                 % automatic line breaking only at whitespace
    captionpos=b,                    % sets the caption-position to bottom
    commentstyle=\color{mygreen},    % comment style
    escapeinside={(*@}{@*)},         % if you want to add LaTeX within your code
    keywordstyle=\color{blue},       % keyword style
    stringstyle=\color{mymauve},     % string literal style
    tabsize=1,
    % where to put the line-numbers; possible values are (none, left, right)
    numbers = left,
    % how far the line-numbers are from the code
    numbersep = 10 pt,
    % the style that is used for the line-numbers
    numberstyle = \ttfamily,
    % the step between two line-numbers. If it's 1, each line will be numbered 
    stepnumber = 1,
}
\usepackage{ulem}
\usepackage{mathtools}
\begin{document}
\maketitle

\section*{Extra Credit}
(25 points) Prove that the system resulting from the composition of two restrictive systems is itself restrictive.

\subsubsection*{Solution}

Note: This proof borrowed ideas from the work that proposed restrictiveness: \textit{Noninterference and the Composability of Security Properties}[1]

\begin{proof}
    Suppose we have two two-bit machines $A$ and $B$, $A$ and $B$ are composed together.
    \begin{enumerate}
        \item Given input, $A$ and $B$ must be in a different state, causing the composed machine to be in a different state.
        
        \item Since given HIGH/LOG input sequence, $A$ and $B$ has such equivalance states, the composed machine does too.
        
        \item Suppose composed machine has equivalance states $\sigma_i \equiv\sigma_j$. 
        That means states in $A$ and $B$ are equivalt too, i.e. $\sigma^A_i \equiv \sigma^A_j$, $\sigma^B_i \equiv \sigma^B_j$.
        Therefore, exists a sequence of HIGH inputs that cause $\sigma^A_i$ and $\sigma^A_j$. 
        Similarly there is a sequence of HIGH inputs for $B$, part of which are outputs of $A$.
        Therefore, we can compose these inputs as necessary to generate two sequences of inputs for the composed machine.
        
        \item Suppose composed machine has equivalance states $\sigma_i \equiv\sigma_j$. 
        We have LOW input $e_A$, $e_B$ for machine $A$ and $B$.
        That means states in $A$ and $B$ are equivalt too, i.e. $\sigma^A_i \equiv \sigma^A_j$, $\sigma^B_i \equiv \sigma^B_j$.
        Thus we have sequence of $c_Ae_Ad_A$ and $c_Be_Bd_B$ that transfers $A$ from $\sigma^A_i$ to $\sigma'^A_i$, $B$ from $\sigma^B_i$ to $\sigma'^B_i$.
        We can also find sequences of $c'_Ae_Ad'_A$ and $c'_Be_Bd'_B$ since $A$ and $B$ are restrictive.
        
        Part of $c_Be_Bd_B$ belongs to the output of $A$, therefore, we can compose $c_Ae_Ad_A$ and $c_Be_Bd_B$ to a new sequence $ce_Ad$.
        We can compose $c'_Ae_Ad'_A$ and $c'_Be_Bd'_B$ to a new sequence $c'ed'$.
        
        Thus property 4 is satisfied.
    \end{enumerate}
\end{proof}


[1] D. McCullough. “Noninterference and the Composability of Security Properties,” Proceedings of the 1988 IEEE Symposium on Security and Privacy pp. 177–186 (Apr. 1988)

\end{document}
\grid
