\section*{Problem 3}

 (29 points) Theorem 3.1, used in the proof of the HRU result, states:
``Suppose two subjects $s_1$ and $s_2$ are created and the rights in $A[s_1, o_1]$ and $A[s_2, o_2]$ are tested.
The same test for $A[s_1, o_1]$ and $A[s_1, o_2] = A[s_1, o_2] \cup A[s_2, o_2]$ will produce the same result.''
Justify this statement.
Would it be true if one could test for the absence of rights as well as for the presence of rights?

\subsubsection*{Solution}

Since set union is equivalent to boolean or, we can have:

\begin{align*}
    have\_right(A[s_1, o_2]) & = have\_right(A[s_1, o_2]) \textbf{ or } have\_right([s_2, o_2]) \\
                             & = have\_right(A[s_1, o_2] \cup A[s_2, o_2])
\end{align*}

But if we can test the absence, the logic wouldn't work.

When merging and testing $have\_right$, any subject have the right would casue the merged subject has the right.
But when testing $no\_right$, we have to make sure all the subjects merged have no right, therefore, using union wouldn't work.
We should use intersection instead.

We have to write the equation like the following:

\begin{align*}
    no\_right(A[s_1, o_2]) & = no\_right(A[s_1, o_2])  \textbf{ and }  no\_right([s_2, o_2]) \\
                           & = no\_right(A[s_1, o_2] \cap A[s_2, o_2])
\end{align*}